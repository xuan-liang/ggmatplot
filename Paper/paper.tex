\documentclass[10pt,a4paper,onecolumn]{article}
\usepackage{marginnote}
\usepackage{graphicx}
\usepackage{xcolor}
\usepackage{authblk,etoolbox}
\usepackage{titlesec}
\usepackage{calc}
\usepackage{tikz}
\usepackage{hyperref}
\hypersetup{colorlinks,breaklinks,
            urlcolor=[rgb]{0.0, 0.5, 1.0},
            linkcolor=[rgb]{0.0, 0.5, 1.0}}
\usepackage{caption}
\usepackage{tcolorbox}
\usepackage{amssymb,amsmath}
\usepackage{ifxetex,ifluatex}
\usepackage{seqsplit}
\usepackage{fixltx2e} % provides \textsubscript
\usepackage[
  backend=biber,
%  style=alphabetic,
%  citestyle=numeric
]{biblatex}
\bibliography{paper.bib}


% --- Page layout -------------------------------------------------------------
\usepackage[top=3.5cm, bottom=3cm, right=1.5cm, left=1.0cm,
            headheight=2.2cm, reversemp, includemp, marginparwidth=4.5cm]{geometry}

% --- Default font ------------------------------------------------------------
% \renewcommand\familydefault{\sfdefault}

% --- Style -------------------------------------------------------------------
\renewcommand{\bibfont}{\small \sffamily}
\renewcommand{\captionfont}{\small\sffamily}
\renewcommand{\captionlabelfont}{\bfseries}

% --- Section/SubSection/SubSubSection ----------------------------------------
\titleformat{\section}
  {\normalfont\sffamily\Large\bfseries}
  {}{0pt}{}
\titleformat{\subsection}
  {\normalfont\sffamily\large\bfseries}
  {}{0pt}{}
\titleformat{\subsubsection}
  {\normalfont\sffamily\bfseries}
  {}{0pt}{}
\titleformat*{\paragraph}
  {\sffamily\normalsize}


% --- Header / Footer ---------------------------------------------------------
\usepackage{fancyhdr}
\pagestyle{fancy}
\fancyhf{}
%\renewcommand{\headrulewidth}{0.50pt}
\renewcommand{\headrulewidth}{0pt}
\fancyhead[L]{\hspace{-0.75cm}\includegraphics[width=5.5cm]{/Library/Frameworks/R.framework/Versions/4.1/Resources/library/rticles/rmarkdown/templates/joss/resources/JOSS-logo.png}}
\fancyhead[C]{}
\fancyhead[R]{}
\renewcommand{\footrulewidth}{0.25pt}

\fancyfoot[L]{\footnotesize{\sffamily Liang et.
al., (2022). \texttt{ggmatplot}: An R package for data visualization on
wide-format
data. \textit{Journal of Open Source Software}, (), . \href{https://doi.org/}{https://doi.org/}}}


\fancyfoot[R]{\sffamily \thepage}
\makeatletter
\let\ps@plain\ps@fancy
\fancyheadoffset[L]{4.5cm}
\fancyfootoffset[L]{4.5cm}

% --- Macros ---------

\definecolor{linky}{rgb}{0.0, 0.5, 1.0}

\newtcolorbox{repobox}
   {colback=red, colframe=red!75!black,
     boxrule=0.5pt, arc=2pt, left=6pt, right=6pt, top=3pt, bottom=3pt}

\newcommand{\ExternalLink}{%
   \tikz[x=1.2ex, y=1.2ex, baseline=-0.05ex]{%
       \begin{scope}[x=1ex, y=1ex]
           \clip (-0.1,-0.1)
               --++ (-0, 1.2)
               --++ (0.6, 0)
               --++ (0, -0.6)
               --++ (0.6, 0)
               --++ (0, -1);
           \path[draw,
               line width = 0.5,
               rounded corners=0.5]
               (0,0) rectangle (1,1);
       \end{scope}
       \path[draw, line width = 0.5] (0.5, 0.5)
           -- (1, 1);
       \path[draw, line width = 0.5] (0.6, 1)
           -- (1, 1) -- (1, 0.6);
       }
   }

% --- Title / Authors ---------------------------------------------------------
% patch \maketitle so that it doesn't center
\patchcmd{\@maketitle}{center}{flushleft}{}{}
\patchcmd{\@maketitle}{center}{flushleft}{}{}
% patch \maketitle so that the font size for the title is normal
\patchcmd{\@maketitle}{\LARGE}{\LARGE\sffamily}{}{}
% patch the patch by authblk so that the author block is flush left
\def\maketitle{{%
  \renewenvironment{tabular}[2][]
    {\begin{flushleft}}
    {\end{flushleft}}
  \AB@maketitle}}
\makeatletter
\renewcommand\AB@affilsepx{ \protect\Affilfont}
%\renewcommand\AB@affilnote[1]{{\bfseries #1}\hspace{2pt}}
\renewcommand\AB@affilnote[1]{{\bfseries #1}\hspace{3pt}}
\makeatother
\renewcommand\Authfont{\sffamily\bfseries}
\renewcommand\Affilfont{\sffamily\small\mdseries}
\setlength{\affilsep}{1em}


\ifnum 0\ifxetex 1\fi\ifluatex 1\fi=0 % if pdftex
  \usepackage[T1]{fontenc}
  \usepackage[utf8]{inputenc}

\else % if luatex or xelatex
  \ifxetex
    \usepackage{mathspec}
  \else
    \usepackage{fontspec}
  \fi
  \defaultfontfeatures{Ligatures=TeX,Scale=MatchLowercase}

\fi
% use upquote if available, for straight quotes in verbatim environments
\IfFileExists{upquote.sty}{\usepackage{upquote}}{}
% use microtype if available
\IfFileExists{microtype.sty}{%
\usepackage{microtype}
\UseMicrotypeSet[protrusion]{basicmath} % disable protrusion for tt fonts
}{}

\usepackage{hyperref}
\hypersetup{unicode=true,
            pdftitle={ggmatplot: An R package for data visualization on wide-format data},
            pdfborder={0 0 0},
            breaklinks=true}
\urlstyle{same}  % don't use monospace font for urls
\usepackage{color}
\usepackage{fancyvrb}
\newcommand{\VerbBar}{|}
\newcommand{\VERB}{\Verb[commandchars=\\\{\}]}
\DefineVerbatimEnvironment{Highlighting}{Verbatim}{commandchars=\\\{\}}
% Add ',fontsize=\small' for more characters per line
\usepackage{framed}
\definecolor{shadecolor}{RGB}{248,248,248}
\newenvironment{Shaded}{\begin{snugshade}}{\end{snugshade}}
\newcommand{\AlertTok}[1]{\textcolor[rgb]{0.94,0.16,0.16}{#1}}
\newcommand{\AnnotationTok}[1]{\textcolor[rgb]{0.56,0.35,0.01}{\textbf{\textit{#1}}}}
\newcommand{\AttributeTok}[1]{\textcolor[rgb]{0.77,0.63,0.00}{#1}}
\newcommand{\BaseNTok}[1]{\textcolor[rgb]{0.00,0.00,0.81}{#1}}
\newcommand{\BuiltInTok}[1]{#1}
\newcommand{\CharTok}[1]{\textcolor[rgb]{0.31,0.60,0.02}{#1}}
\newcommand{\CommentTok}[1]{\textcolor[rgb]{0.56,0.35,0.01}{\textit{#1}}}
\newcommand{\CommentVarTok}[1]{\textcolor[rgb]{0.56,0.35,0.01}{\textbf{\textit{#1}}}}
\newcommand{\ConstantTok}[1]{\textcolor[rgb]{0.00,0.00,0.00}{#1}}
\newcommand{\ControlFlowTok}[1]{\textcolor[rgb]{0.13,0.29,0.53}{\textbf{#1}}}
\newcommand{\DataTypeTok}[1]{\textcolor[rgb]{0.13,0.29,0.53}{#1}}
\newcommand{\DecValTok}[1]{\textcolor[rgb]{0.00,0.00,0.81}{#1}}
\newcommand{\DocumentationTok}[1]{\textcolor[rgb]{0.56,0.35,0.01}{\textbf{\textit{#1}}}}
\newcommand{\ErrorTok}[1]{\textcolor[rgb]{0.64,0.00,0.00}{\textbf{#1}}}
\newcommand{\ExtensionTok}[1]{#1}
\newcommand{\FloatTok}[1]{\textcolor[rgb]{0.00,0.00,0.81}{#1}}
\newcommand{\FunctionTok}[1]{\textcolor[rgb]{0.00,0.00,0.00}{#1}}
\newcommand{\ImportTok}[1]{#1}
\newcommand{\InformationTok}[1]{\textcolor[rgb]{0.56,0.35,0.01}{\textbf{\textit{#1}}}}
\newcommand{\KeywordTok}[1]{\textcolor[rgb]{0.13,0.29,0.53}{\textbf{#1}}}
\newcommand{\NormalTok}[1]{#1}
\newcommand{\OperatorTok}[1]{\textcolor[rgb]{0.81,0.36,0.00}{\textbf{#1}}}
\newcommand{\OtherTok}[1]{\textcolor[rgb]{0.56,0.35,0.01}{#1}}
\newcommand{\PreprocessorTok}[1]{\textcolor[rgb]{0.56,0.35,0.01}{\textit{#1}}}
\newcommand{\RegionMarkerTok}[1]{#1}
\newcommand{\SpecialCharTok}[1]{\textcolor[rgb]{0.00,0.00,0.00}{#1}}
\newcommand{\SpecialStringTok}[1]{\textcolor[rgb]{0.31,0.60,0.02}{#1}}
\newcommand{\StringTok}[1]{\textcolor[rgb]{0.31,0.60,0.02}{#1}}
\newcommand{\VariableTok}[1]{\textcolor[rgb]{0.00,0.00,0.00}{#1}}
\newcommand{\VerbatimStringTok}[1]{\textcolor[rgb]{0.31,0.60,0.02}{#1}}
\newcommand{\WarningTok}[1]{\textcolor[rgb]{0.56,0.35,0.01}{\textbf{\textit{#1}}}}
\usepackage{graphicx,grffile}
\makeatletter
\def\maxwidth{\ifdim\Gin@nat@width>\linewidth\linewidth\else\Gin@nat@width\fi}
\def\maxheight{\ifdim\Gin@nat@height>\textheight\textheight\else\Gin@nat@height\fi}
\makeatother
% Scale images if necessary, so that they will not overflow the page
% margins by default, and it is still possible to overwrite the defaults
% using explicit options in \includegraphics[width, height, ...]{}
\setkeys{Gin}{width=\maxwidth,height=\maxheight,keepaspectratio}
\IfFileExists{parskip.sty}{%
\usepackage{parskip}
}{% else
\setlength{\parindent}{0pt}
\setlength{\parskip}{6pt plus 2pt minus 1pt}
}
\setlength{\emergencystretch}{3em}  % prevent overfull lines
\providecommand{\tightlist}{%
  \setlength{\itemsep}{0pt}\setlength{\parskip}{0pt}}
\setcounter{secnumdepth}{0}
% Redefines (sub)paragraphs to behave more like sections
\ifx\paragraph\undefined\else
\let\oldparagraph\paragraph
\renewcommand{\paragraph}[1]{\oldparagraph{#1}\mbox{}}
\fi
\ifx\subparagraph\undefined\else
\let\oldsubparagraph\subparagraph
\renewcommand{\subparagraph}[1]{\oldsubparagraph{#1}\mbox{}}
\fi

% Pandoc citation processing
\newlength{\csllabelwidth}
\setlength{\csllabelwidth}{3em}
\newlength{\cslhangindent}
\setlength{\cslhangindent}{1.5em}
% for Pandoc 2.8 to 2.10.1
\newenvironment{cslreferences}%
  {}%
  {\par}
% For Pandoc 2.11+
\newenvironment{CSLReferences}[2] % #1 hanging-ident, #2 entry spacing
 {% don't indent paragraphs
  \setlength{\parindent}{0pt}
  % turn on hanging indent if param 1 is 1
  \ifodd #1 \everypar{\setlength{\hangindent}{\cslhangindent}}\ignorespaces\fi
  % set entry spacing
  \ifnum #2 > 0
  \setlength{\parskip}{#2\baselineskip}
  \fi
 }%
 {}
\usepackage{calc} % for calculating minipage widths
\newcommand{\CSLBlock}[1]{#1\hfill\break}
\newcommand{\CSLLeftMargin}[1]{\parbox[t]{\csllabelwidth}{#1}}
\newcommand{\CSLRightInline}[1]{\parbox[t]{\linewidth - \csllabelwidth}{#1}\break}
\newcommand{\CSLIndent}[1]{\hspace{\cslhangindent}#1}

\usepackage{booktabs}
\usepackage{longtable}
\usepackage{array}
\usepackage{multirow}
\usepackage{wrapfig}
\usepackage{float}
\usepackage{colortbl}
\usepackage{pdflscape}
\usepackage{tabu}
\usepackage{threeparttable}
\usepackage{threeparttablex}
\usepackage[normalem]{ulem}
\usepackage{makecell}
\usepackage{xcolor}

\title{\texttt{ggmatplot}: An R package for data visualization on
wide-format data}

        \author[1]{Xuan Liang}
          \author[1]{Francis K. C. Hui}
          \author[2]{Dilinie Seimon}
          \author[2]{Emi Tanaka}
    
      \affil[1]{Research School of Finance, Actuarial Studies and
Statistics, The Australian National University}
      \affil[2]{Department of Econometrics and Business Statistics,
Monash University}
  \date{\vspace{-5ex}}

\begin{document}
\maketitle

\marginpar{
  %\hrule
  \sffamily\small

  {\bfseries DOI:} \href{https://doi.org/}{\color{linky}{}}

  \vspace{2mm}

  {\bfseries Software}
  \begin{itemize}
    \setlength\itemsep{0em}
    \item \href{}{\color{linky}{Review}} \ExternalLink
    \item \href{}{\color{linky}{Repository}} \ExternalLink
    \item \href{}{\color{linky}{Archive}} \ExternalLink
  \end{itemize}

  \vspace{2mm}

  {\bfseries Submitted:} \\
  {\bfseries Published:} 

  \vspace{2mm}
  {\bfseries License}\\
  Authors of papers retain copyright and release the work under a Creative Commons Attribution 4.0 International License (\href{http://creativecommons.org/licenses/by/4.0/}{\color{linky}{CC-BY}}).
}

\hypertarget{summary}{%
\section{Summary}\label{summary}}

The layered grammar of graphics (H. Wickham, 2010), implemented as the
\texttt{ggplot2} package (Hadley Wickham, 2016) in the statistical
language R (R Core Team, 2021), is a powerful and popular tool to create
versatile statistical graphics. This graphical system, however, requires
input data to be organised in a manner that a data column is mapped to
an aesthetic element (e.g.~x-coordinate, y-coordinate, color, size),
which create friction in constructing plots with an aesthetic element
that span multiple columns in the original data by requiring users to
re-organise the data.

The \texttt{ggmatplot}, built upon \texttt{ggplot2}, is an R-package
that allows quick plotting across the columns of matrices or data with
the result returned as a \texttt{ggplot} object. The package is inspired
by the function \texttt{matplot()} in the core R \texttt{graphics}
system, thus \texttt{ggmatplot} can be considered as a \texttt{ggplot}
version of \texttt{matplot} with the benefits of customising the plots
as any other \texttt{ggplot} objects via \texttt{ggplot2} functions.

\hypertarget{statement-of-need}{%
\section{Statement of need}\label{statement-of-need}}

Input data to construct plots with \texttt{ggplot2} require data to be
organised in a manner that maps data columns to aesthetic elements. This
generally works well where data is tidied in a rectangular form,
referred to as ``tidy data'' (Hadley Wickham, 2014), where each row
represents an observational unit, each column represents a variable, and
each cell represents a value. In some cases, what constitutes a variable
(or observational unit), hence a column (or row), in a tidy data can be
dependent upon interpretation or downstream interest (e.g.~Tables
\ref{tab:tab1} and \ref{tab:tab2} can be both considered as tidy data),
but a clear violation of tidy data principles is when the column names
contain data values, e.g.~Table \ref{tab:tab3} contain the name of the
species across a number of column names.

\begin{table}

\caption{\label{tab:tab1}Restaurant rating data in "tidy" form. The first column shows the restaurant ID, and the next four columns show the average ratings (out of 5) for food, service, ambience and overall, respectively.}
\centering
\begin{tabular}[t]{lrrrr}
\toprule
\multicolumn{1}{c}{ } & \multicolumn{4}{c}{Rating} \\
\cmidrule(l{3pt}r{3pt}){2-5}
Restaurant & Food & Service & Ambience & Overall\\
\midrule
R1 & 4 & 3 & 4 & 4\\
R2 & 4 & 5 & 4 & 4\\
R3 & 3 & 4 & 5 & 3\\
R4 & 2 & 4 & 4 & 3\\
R5 & 3 & 4 & 4 & 3\\
\bottomrule
\end{tabular}
\end{table}

\begin{table}

\caption{\label{tab:tab2}Another form for the restaurant rating data in Table \ref{tab:tab1}. In @Wickham2014-gy, this format is called the "molten" data.}
\centering
\begin{tabular}[t]{llr}
\toprule
Restauant & Rating type & Rating\\
\midrule
R1 & food & 4\\
R1 & service & 3\\
R1 & ambience & 4\\
R1 & overall & 4\\
R2 & food & 4\\
R2 & service & 5\\
R2 & ambience & 4\\
R2 & overall & 4\\
R3 & food & 3\\
R3 & service & 4\\
R3 & ambience & 5\\
R3 & overall & 3\\
R4 & food & 2\\
R4 & service & 4\\
R4 & ambience & 4\\
R4 & overall & 3\\
R5 & food & 3\\
R5 & service & 4\\
R5 & ambience & 4\\
R5 & overall & 3\\
\bottomrule
\end{tabular}
\end{table}

\begin{table}

\caption{\label{tab:tab3}Spider abundance data with environmental covariates. The rows correspond to the site, the first two columns are environmental covariates that measure the soil dry mass and cover moss, and the following five columns shows the abundance of the species.}
\centering
\begin{tabular}[t]{rrrrrrrr}
\toprule
\multicolumn{1}{c}{ } & \multicolumn{2}{c}{Environment covariates} & \multicolumn{5}{c}{Species abundance} \\
\cmidrule(l{3pt}r{3pt}){2-3} \cmidrule(l{3pt}r{3pt}){4-8}
Site & Soil dry mass & Moss & Alopcune & Arctlute & Pardpull & Trocterr & Zoraspin\\
\midrule
1 & 2.3321 & 3.0445 & 10 & 0 & 45 & 57 & 4\\
2 & 3.0493 & 1.0986 & 2 & 0 & 37 & 65 & 9\\
3 & 2.5572 & 2.3979 & 20 & 0 & 45 & 66 & 1\\
\bottomrule
\end{tabular}
\end{table}

The organisation of the data is largely dependent on the downstream
analysis and there is no one correct way to do this. Some forms of
multivariate data, e.g.~Table \ref{tab:tab3}, are prevalent in the field
because it aligns as an input data for a modelling software and/or the
format is more convenient for input or view of the data in spreadsheet
format. However, this format is not consistent with the required format
for \texttt{ggplot2}, and consequently, plotting with \texttt{ggplot2}
interrupts the workflow of a user that is trying to quickly visualise
these types of data. The \texttt{ggmatplot} R-package provides a
solution to this common friction in producing plots with
\texttt{ggplot2}.

\hypertarget{examples}{%
\section{Examples}\label{examples}}

In this section we demonstrate the use of the \texttt{ggmatplot} package
and contrast the specification with \texttt{ggplot2} with data wrangling
using \texttt{dplyr} and \texttt{tidyr} (Hadley Wickham et al., 2019)
using the example data in Tables \ref{tab:tab1} and \ref{tab:tab3},
which are stored in the objects \texttt{wide\_df} and \texttt{abun\_df},
respectively.

\hypertarget{example-1}{%
\subsection{Example 1}\label{example-1}}

\begin{Shaded}
\begin{Highlighting}[]
\FunctionTok{library}\NormalTok{(ggmatplot)}
\FunctionTok{ggmatplot}\NormalTok{(}\AttributeTok{x =}\NormalTok{ wide\_df[, }\SpecialCharTok{{-}}\DecValTok{1}\NormalTok{], }\AttributeTok{plot\_type =} \StringTok{"both"}\NormalTok{,}
          \AttributeTok{xlab =} \StringTok{"Restaurant"}\NormalTok{,  }\AttributeTok{ylab =} \StringTok{"Rating"}\NormalTok{, }\AttributeTok{legend\_title =} \StringTok{"Type"}\NormalTok{)}
\end{Highlighting}
\end{Shaded}

\includegraphics{paper_files/figure-latex/matplot1-1.pdf}

\begin{Shaded}
\begin{Highlighting}[]
\FunctionTok{library}\NormalTok{(ggplot2)}
\FunctionTok{library}\NormalTok{(tidyr) }\CommentTok{\# or library(tidyverse)}
\NormalTok{wide\_df }\SpecialCharTok{\%\textgreater{}\%} 
  \FunctionTok{pivot\_longer}\NormalTok{(}\FunctionTok{contains}\NormalTok{(}\StringTok{"rating"}\NormalTok{), }
               \AttributeTok{names\_to =} \StringTok{"rating\_type"}\NormalTok{,}
               \AttributeTok{values\_to =} \StringTok{"rating"}\NormalTok{) }\SpecialCharTok{\%\textgreater{}\%} 
  \FunctionTok{ggplot}\NormalTok{(}\FunctionTok{aes}\NormalTok{(restaurant, rating, }\AttributeTok{color =}\NormalTok{ rating\_type)) }\SpecialCharTok{+} 
  \FunctionTok{geom\_point}\NormalTok{() }\SpecialCharTok{+}
  \FunctionTok{geom\_line}\NormalTok{(}\FunctionTok{aes}\NormalTok{(}\AttributeTok{group =}\NormalTok{ rating\_type,}
                \AttributeTok{linetype =}\NormalTok{ rating\_type))}
\end{Highlighting}
\end{Shaded}

\hypertarget{example-2}{%
\subsection{Example 2}\label{example-2}}

\begin{Shaded}
\begin{Highlighting}[]
\FunctionTok{ggmatplot}\NormalTok{(}\AttributeTok{x =}\NormalTok{ wide\_df[, }\DecValTok{2}\SpecialCharTok{:}\DecValTok{4}\NormalTok{], }\AttributeTok{y =}\NormalTok{ wide\_df[, }\DecValTok{5}\NormalTok{], }\AttributeTok{plot\_type =} \StringTok{"both"}\NormalTok{)}
\end{Highlighting}
\end{Shaded}

\includegraphics{paper_files/figure-latex/matplot2-1.pdf}

\hypertarget{discussion}{%
\section{Discussion}\label{discussion}}

The \texttt{ggmatplot} R-package provides a solution to a common
friction to quickly plotting multivariate data where the primary
interest is mapping the column names as an aesthetic element. The
solution provided however is a recipe-driven approach where the user can
only produce plot types as many there are included in the
\texttt{plot\_type} option. Future development of the package could
benefit from using a grammar approach, like in Wilkinson (2005) and H.
Wickham (2010), where plot types can be extensible.

\hypertarget{acknowledgements}{%
\section{Acknowledgements}\label{acknowledgements}}

FKCH was supported by ARC DECRA XXX.

\hypertarget{references}{%
\section*{References}\label{references}}
\addcontentsline{toc}{section}{References}

\hypertarget{refs}{}
\begin{CSLReferences}{1}{0}
\leavevmode\vadjust pre{\hypertarget{ref-rstats}{}}%
R Core Team. (2021). \emph{R: A language and environment for statistical
computing}. Vienna, Austria: R Foundation for Statistical Computing.
Retrieved from \url{https://www.R-project.org/}

\leavevmode\vadjust pre{\hypertarget{ref-Wickham2010-kt}{}}%
Wickham, H. (2010). A layered grammar of graphics. \emph{Journal of
computational and graphical statistics: a joint publication of American
Statistical Association, Institute of Mathematical Statistics, Interface
Foundation of North America}.

\leavevmode\vadjust pre{\hypertarget{ref-Wickham2014-gy}{}}%
Wickham, Hadley. (2014). Tidy data. \emph{Journal of Statistical
Software}, \emph{59}(10), 1--23.
doi:\href{https://doi.org/10.18637/jss.v059.i10}{10.18637/jss.v059.i10}

\leavevmode\vadjust pre{\hypertarget{ref-Wickham2016}{}}%
Wickham, Hadley. (2016). \emph{ggplot2: Elegant graphics for data
analysis}. Springer-Verlag New York. Retrieved from
\url{https://ggplot2.tidyverse.org}

\leavevmode\vadjust pre{\hypertarget{ref-Wickham2019}{}}%
Wickham, Hadley, Averick, M., Bryan, J., Chang, W., McGowan, L. D.,
François, R., Grolemund, G., et al. (2019). Welcome to the {tidyverse}.
\emph{Journal of Open Source Software}, \emph{4}(43), 1686.
doi:\href{https://doi.org/10.21105/joss.01686}{10.21105/joss.01686}

\leavevmode\vadjust pre{\hypertarget{ref-Wilkinson2005-oz}{}}%
Wilkinson, L. (2005). \emph{The grammar of graphics}. Springer.

\end{CSLReferences}

\end{document}
